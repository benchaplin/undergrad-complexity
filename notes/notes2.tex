\documentclass[11pt]{article}
\usepackage{amsmath, amsthm, amssymb}
\usepackage[margin=0.75in]{geometry}

\theoremstyle{plain}
\newtheorem{thm}{Theorem}
\newtheorem{cor}{Corollary}
\newtheorem{lem}{Lemma}
\theoremstyle{definition}
\newtheorem*{defn}{Definition}
\newtheorem*{ex}{Example}

\newcommand{\C}{\mathbb{C}}
\newcommand{\R}{\mathbb{R}}
\newcommand{\Q}{\mathbb{Q}}
\newcommand{\Z}{\mathbb{Z}}
\newcommand{\N}{\mathbb{N}}

\title{Undergrad Complexity Theory: Notes 2}
\author{Ben Chaplin}
\date{}

\begin{document}

\maketitle
\tableofcontents

\section{Background}
\subsection{Languages}

\begin{defn}
    A {\bf language L} over an alphabet $\Sigma$ is a subset of strings $L \subseteq \Sigma^*$.
\end{defn}

\begin{ex}
PRIMES $= \{\langle x \rangle : x \in \N, x$ is prime$\}$, equivalent to the decision problem $\{0, 1\}^* \rightarrow \{$no, yes$\}$.
\end{ex}

\subsection{Algorithms}

What is an algorithm? We can define the concept using programming languages, or better, models of computation. Turing machines are 
a good choice as they're easy to formalize. Furthermore: \medskip

\noindent {\bf Church-Turing Thesis}. Any real-world algorithm can be simulated by Turing Machines.\medskip

However, this is a statement about computability. In terms of complexity theory, it's important to note that an algorithm running in 
time $T$ in C-like pseudocode, it can compiled to a Turing machine running in time roughly $T^4$. 

\section{Turing Machines}
\subsection{Model}

Official model of a Turing Machine: 
\begin{itemize}
    \item one tape
    \item two-way infinite
\end{itemize}

Roughly, the tape holds symbols from an alphabet and reads/writes with a "head" pointer. The source code tells us how to move and write.

\subsection{Formal definition}

\begin{defn}
    A {\bf Turing machine M} is a 7-tuple $M = (\Gamma, b, \Sigma, Q, q_0, F, \delta)$, where
    \begin{itemize}
        \item $\Gamma$ is a finite, non-empty set of tape symbols
        \item $b \in \Gamma$ is the blank symbol (allowed to occur infinitely on tape)
        \item $\Sigma \subseteq (\Gamma - \{b\})$ is the input alphabet 
        \item $Q$ is a finite, non-empty set of states
        \item $q_0$ is the initial state 
        \item $F \subseteq Q$ is the set of accepting states 
        \item $\delta: (Q - F) \times \Gamma \nrightarrow Q \times \Gamma \times \{$left, right$\}$ is the transition function. If $\delta$ 
            is not defined on the current tape symbol, the machine halts.
    \end{itemize}
\end{defn}









\end{document}

