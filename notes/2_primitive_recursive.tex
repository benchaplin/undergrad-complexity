\documentclass[11pt]{article}
\usepackage{amsmath, amsthm, amssymb}
\usepackage[margin=0.75in]{geometry}
\usepackage{pict2e,amsmath,relsize}

\DeclareRobustCommand{\xvdash}[2][]{\mathrel{\drawxvdash{#1}{#2}}}

\newcommand{\drawxvdash}[2]{%
  \vcenter{\hbox{%
    \setlength{\unitlength}{1em}%
    \begin{picture}(1,1)
    \roundcap
    \put(0,0){\line(0,1){1}}
    \put(0,0.5){\line(1,0){1}}
    \put(0.5,0){\makebox[0pt]{\text{\smaller$\scriptscriptstyle#2$}}}
    \put(0.5,0.6){\makebox[0pt]{\text{\smaller$\scriptscriptstyle#1$}}}
    \end{picture}%
  }}%
}

\theoremstyle{plain}
\newtheorem{thm}{Theorem}
\newtheorem{cor}{Corollary}
\newtheorem{lem}{Lemma}
\theoremstyle{definition}
\newtheorem*{defn}{Definition}
\newtheorem*{ex}{Example}

\newcommand{\C}{\mathbb{C}}
\newcommand{\R}{\mathbb{R}}
\newcommand{\Q}{\mathbb{Q}}
\newcommand{\Z}{\mathbb{Z}}
\newcommand{\N}{\mathbb{N}}

\title{Undergrad Complexity}
\author{Ben Chaplin}
\date{}

\begin{document}

\maketitle
\tableofcontents

\section{Primitive Recursive Functions}
\subsection{Various functions and operators}

\begin{defn}
	The {\bf constant function $C^k_n$} is defined for all $n, k \in \N$ as $C^k_n(x_1, \ldots, x_k) = n$.
\end{defn}

\begin{defn}
	The {\bf successor function $S$} is defined for all $x \in N$ as $S(x) = x + 1$.
\end{defn}

\begin{defn}
	The {\bf projection function $P^k_i$} is defined for all $i, k \in \N$, $1 \leq i \leq k$ as
	$P^k_i(x_1, \ldots, x_k) = x_i$.
\end{defn}

\begin{defn}
	Given an $m$-ary function $h$ and $m$ $k$-ary functions $g_1, \ldots, g_m$, the {\bf composition operation $\circ$}
	is defined:
	\begin{align*}
		f                   & = h \circ (g_1, \ldots, g_m)                              \\
		f(x_1, \ldots, x_k) & = h(g_1(x_1, \ldots, x_k), \ldots, g_m(x_1, \ldots, x_k))
	\end{align*}
\end{defn}

\begin{defn}
	Given a $k$-ary function $g$ and a $(k + 2)$-ary function $h$, the {\bf primitive recursion operator $\rho$} is defined:
	\begin{align*}
		f                         & = \rho(g, h)                                     \\
		f(0, x_1, \ldots, x_k)    & = g(x_1, \ldots, x_k)                            \\
		f(S(y), x_1, \ldots, x_k) & = h(y, f(y, x_1, \ldots, x_k), x_1, \ldots, x_k)
	\end{align*}
\end{defn}

\subsection{Definition and examples}

\begin{defn}
	The constant function, successor function and projection function are {\bf primitive recursive functions}, as well as
	any finite number of composition or primitive recursive operations on those functions.
\end{defn}

\begin{ex}
    $Add: \N^2 \rightarrow \N$ is primitive recursive, as it can be defined by:
    $Add = \rho(P^1_1, S \circ P^3_2)$.

    \begin{align*}
        Add(0, x) &= P^1_1(x) \\
        Add(S(y), x) &= S \circ P^3_2(y, f(y, x), x)
    \end{align*}

    Let's run $Add(2, 3)$:
    \begin{align*}
        Add(2, 3) &= \rho(P^1_1, S\circ P^3_2)(S(1), 3) \\
                  &= (S\circ P^3_2)(1, Add(1, 3), 3)\\
                  &= S(Add(1, 3))\\
                  &= S(\rho(P^1_1, S\circ P^3_2)(S(0), 3))\\
                  &= S((S\circ P^3_2)(0, Add(0, 3), 3))\\
                  &= S(S(Add(0, 3)))\\
                  &= S(S(P^1_1(3)))\\
                  &= S(S(3))\\
                  &= S(4)\\
                  &= 5
    \end{align*}
\end{ex}

\begin{ex}
    $Mult: \N^2 \rightarrow \N$ is primitive recursive, as it can be defined by:
    $Mult = \rho(C^1_0, Add \circ (P^3_2, P^3_3))$.

    \begin{align*}
        Mult(0, x) &= C^1_0 \\
        Mult(S(y), x) &= Add \circ (P^3_2, P^3_3)(y, Mult(y, x), x)
    \end{align*}
\end{ex}

\subsection{Computability}

It's pretty clear that all primitive recursive functions are computable, but are there computable functions that 
aren't primitive recursive?

\begin{thm}
    There exists a computable function which is not primitive recursive.
\end{thm}
\begin{proof}
    Roughly: enumerate primitive recursive functions, diagonalization argument.
\end{proof}
















\end{document}

