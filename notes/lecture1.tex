\documentclass[11pt]{article}
\usepackage{amsmath, amsthm, amssymb}
\usepackage[margin=0.75in]{geometry}

\theoremstyle{plain}
\newtheorem{thm}{Theorem}
\newtheorem{cor}{Corollary}
\newtheorem{lem}{Lemma}
\theoremstyle{definition}
\newtheorem*{defn}{Definition}
\newtheorem*{ex}{Example}

\newcommand{\C}{\mathbb{C}}
\newcommand{\R}{\mathbb{R}}
\newcommand{\Q}{\mathbb{Q}}
\newcommand{\Z}{\mathbb{Z}}
\newcommand{\N}{\mathbb{N}}

\title{Undergrad Complexity: Lecture 1}
\author{Ben Chaplin}
\date{}

\begin{document}

\maketitle
\tableofcontents

\section{Overview}
\subsection{Course details}

{\bf Course Title}: Undergrad Complexity\\
{\bf Teacher}: Professor Ryan O'Donnell\\
{\bf School:} Carnegie Melon University\\
{\bf Lectures}: https://www.youtube.com/playlist?list=PLm3J0oaFux3YL5vLXpzOyJiLtqLp6dCW2\\
{\bf Textbook}: {\it Introduction to the Theory of Computation} by Michael Sipser

\section{Introductory Definitions}
\subsection{Alphabets and Strings}

\begin{defn}
    {\bf Computational tasks} are processes which, given an input, should produce a certain kind of 
    output.
\end{defn}

In general, we encode both inputs and outputs using a given set of characters.

\begin{defn}
    An {\bf alphabet} $\Sigma$ is a non-empty finite set of symbols.
\end{defn}

\begin{ex}
    $\Sigma = \{0, 1\}$.
\end{ex}

\begin{defn}
    $\Sigma^n$ is the set of all {\bf strings} of length exactly $n$ made up of symbols in the
    alphabet $\Sigma$.
\end{defn}

\begin{ex}
    Let $\Sigma = \{0, 1\}$, then $\Sigma^2 = \{00, 01, 10, 11\}$.
\end{ex}

Note that $n = 0$ is allowed, the empty string is denoted as $\epsilon$.

\begin{defn}
    $\Sigma^* = \Sigma^0 \cup \Sigma^1 \cup ...$ is the set of all finite length strings made up of
    symbols in the alphabet $\Sigma$.
\end{defn}

\subsection{Encoding Mathematical Objects}

This is an annoying topic, but necessary for a complete analysis of complexity. 

\begin{defn}
    If $X$ is a mathematical object and $\Sigma$ is an alphabet, then $\langle X \rangle_\Sigma \in
    \Sigma^*$ is an {\bf encoding of $X$} (a unique representation of $x$ using the alphabet $\sigma$).
\end{defn}

In general, we are going to avoid rigorously describe encodings. It will be enough to imagine a 
theoretical ``most sensible'' one.

\subsection{Computational Problems}

There are three categories of computational problems:

\begin{itemize}
    \item {\bf Decision problems:} $f: \Sigma^* \to \{0, 1\}$. Problems for which the answer is 
        either ``yes'' or ``no''.
    
        \begin{ex}
            Is a number prime? Does there exist a path in a given graph?
        \end{ex}

    \item {\bf Function problems:} $f: \Sigma^* \to \Sigma'^*$. Problems for which the answer is 
        another string (not necessarily of the same alphabet).

        \begin{ex}
            Convert a decimal number to binary ($\{0, 1, ..., 9\} \to \{0, 1\}$). 
            Factor a prime.
        \end{ex}

    \item {\bf Search problems:} $f: \Sigma^* \to \{x : x \in \Sigma'^*\}$. 
        Problems for which there may be more than one answer, or no answer at all.

        \begin{ex}
            What are the paths in a given graph?
        \end{ex}
\end{itemize}

We primarly work with decision problems. In most cases, search problems and function problems can be
easily reduced to decision problems, without added complexity.































\end{document}

