\documentclass[11pt]{article}
\usepackage{amsmath, amsthm, amssymb, mathtools}
\usepackage[margin=0.75in]{geometry}
\usepackage{pict2e,amsmath,relsize}

\DeclareRobustCommand{\xvdash}[2][]{\mathrel{\drawxvdash{#1}{#2}}}

\newcommand{\drawxvdash}[2]{%
  \vcenter{\hbox{%
    \setlength{\unitlength}{1em}%
    \begin{picture}(1,1)
    \roundcap
    \put(0,0){\line(0,1){1}}
    \put(0,0.5){\line(1,0){1}}
    \put(0.5,0){\makebox[0pt]{\text{\smaller$\scriptscriptstyle#2$}}}
    \put(0.5,0.6){\makebox[0pt]{\text{\smaller$\scriptscriptstyle#1$}}}
    \end{picture}%
  }}%
}

\theoremstyle{plain}
\newtheorem{thm}{Theorem}
\newtheorem{cor}{Corollary}
\newtheorem{lem}{Lemma}
\theoremstyle{definition}
\newtheorem{prob}{Problem}
\newtheorem*{defn}{Definition}
\newtheorem*{ex}{Example}
\newtheorem*{ans}{Answer}

\newcommand{\C}{\mathbb{C}}
\newcommand{\R}{\mathbb{R}}
\newcommand{\Q}{\mathbb{Q}}
\newcommand{\Z}{\mathbb{Z}}
\newcommand{\N}{\mathbb{N}}

\title{Undergrad Complexity: Problem Set 1}
\author{Ben Chaplin}
\date{}

\begin{document}

\maketitle

\begin{prob}
	A variadic function $f: \N^* \rightarrow \N$ is called a {\bf coding function} if there are ``inverse" functions
	$g: \N \rightarrow \N$ and $h: \N \times \N \rightarrow \N$ such that:

	\begin{align*}
		g(f(a_1, \ldots, a_n))    & = n                    \\
        h(f(a_1, \ldots, a_n), i) & = a_i && 1 \leq i \leq n
	\end{align*}
	for all sequences $a_1, \ldots, a_n$. Thus $g$ determines the length of a sequence and $h$ decodes it back to its elements.
	Moreover, $h$ and $g$ are supposed to be easily computable but let's ignore that for the time being. Now consider the pairing
	function $\pi$ defined by:

	$$\pi(x, y) = \binom{x + y + 1}{2} + x + 1$$

	and define $f$ as follows:

	\begin{align*}
		f(nil)              & = 0                             \\
		f(a)                & = \pi(0, a)                     \\
		f(a_1, \ldots, a_n) & = \pi(f(a_2, \ldots, a_n), a_1)
	\end{align*}

	\begin{enumerate}
		\item Show that $\pi$ is injective.
		\item Show that $f$ is a coding function (make sure to explain what the appropriate decoding functions $g$ and $h$ are).
		\item What whould happen if we replaced $\pi(x, y)$ by $\pi(x, y) -1$? How could you fix the issue?
	\end{enumerate}
\end{prob}

\noindent\rule{\textwidth}{0.5pt}

\begin{ans}[1.1]
	First, note that:
	\begin{align}
		\binom{x + y + 1}{2} & = \frac{(x + y)(x +  y + 1)}{2} \\
		                     & = 1 + 2 + \ldots + (x + y)
	\end{align}
	Take $(x_1, y_1), (x_2, y_2) \in \N \times \N$ such that $\pi(x_1, y_1) = \pi(x_2, y_2)$.
	Assume to the contrary that $x_1 + y_1 \neq x_2 + y_2$, and without loss of generality, suppose that $x_1 + y_1 < x_2 + y_2$.
	\begin{align*}
		\binom{x_1 + y_1 + 1}{2} + x_1 + 1     & = \binom{x_2 + y_2 + 1}{2} + x_2 + 1                                    \\
		\shortintertext{By (1) and (2):}
		1 + 2 + \ldots + (x_1 + y_1) + x_1 + 1 & = 1 + 2 + \ldots + (x_2 + y_2)  + x_2 + 1                       \tag{3} \\
		x_1 - x_2                              & = (1 + 2 + \ldots (x_2 + y_2)) - (1 + 2 + \ldots + (x_1 + y_1))         \\
		                                       & \geq x_2 + y_2                                                          \\
		x_1                                    & \geq 2x_2 + y_2                                                         \\
		x_1 + y_1                              & \geq 2x_2 + y_2 + y_1                                                   \\
		                                       & \geq x_2 + y_2,
	\end{align*}
	a contradiction. Thus $x_1 + y_1 = x_2 + y_2$. Then, by (3), $x_1 = x_2$. So $y_1 = y_2$ and $\pi$ is injective.
\end{ans}

\begin{ans}[1.2]
    \begin{lem}
        $f$ is injective over inputs of the same size.
    \end{lem}
    \begin{proof}
        We proceed by induction over the size of the input. If $n = 1$, then $f(x) = \pi(0, x)$, so $f$ is injective by 
        the fact that $\pi$ is injective. Suppose $f$ is injective for inputs of size $n - 1$ and consider inputs of size $n$:
        $(a_1, \ldots, a_n), (b_1, \ldots, b_n) \in \N^n$ where $f(a_1, \ldots, a_n) = f(b_1, \ldots, b_n)$.

        \begin{align*}
            f(a_1, \ldots, a_n) &= f(b_1, \ldots, b_n)\\
            \pi(f(a_2, \ldots, a_n), a_1) &= \pi(f(b_2, \ldots, b_n), b_1)
        \end{align*}

        So by the fact that $\pi$ is injective, $a_1 = b_1$ and $f(a_2, \ldots, a_n) = f(b_2, \ldots, b_n)$. 
        Thus, $a_i = b_i$ for $1 \leq i \leq n$.
    \end{proof}

    \begin{lem}
        $f(\N^n)$ and $f(\N^m)$ are disjoint for any $n \neq m \in \N$.
    \end{lem}
    \begin{proof}
        Suppose $m < n$ and $f(a_1, \ldots, a_m) = f(b_1, \ldots, b_n)$. Then:

        $$\pi(f(a_2, \ldots, a_m), a_1) = \pi(f(b_2, \ldots, b_n), b_1)$$

        Then $a_1 = b_1$. Furthermore, because $\pi$ is injective:

        $$\pi(f(a_3, \ldots, a_m), a_2) = \pi(f(b_3, \ldots, b_n), b_2)$$

        So $a_2 = b_2$. By the same logic, $a_i = b_i$ for $1 \leq i \leq m$. But then we have:

        \begin{align*}
            \pi(f(nil), a_m) &= \pi(f(b_{m + 1}, \ldots, b_n), b_m)\\
            f(nil) &= f(b_{m + 1}, \ldots, b_n)\\
            0 &= f(b_{m + 1}, \ldots, b_n)
        \end{align*}

        But this cannot be the case because $f(x) > 0$ for all $x \in \N^* - \{nil\}$.
    \end{proof}

    {\bf Lemma 1} and {\bf Lemma 2} show that $f$ is injective over $\N^*$. Thus, $f$ has a partial inverse. We can now define
    $g$ and $h$ over all $x \in f(\N^*)$.

    \begin{align*}
        g(x) &= |(f^{-1}(x)|\\
        h(x, i) &= P^{|f^{-1}(x)|}_i(f^{-1}(x)) && \text{where } P \text{ is the projection function}
    \end{align*}
\end{ans}

\begin{ans}[1.3]
    Defining $\pi$ as:

	$$\pi(x, y) = \binom{x + y + 1}{2} + x$$

    does not change the fact that $\pi$ is injective. However, it throws a wrench in our proof of {\bf Lemma 2} from 
    {\bf Answer 1.2}. We can no longer assume that $f(x) > 0$ for all $x \in \N^* - \{nil\}$, as now:

    $$f(0) = f(0, 0) = \ldots = 0$$

    Actually, now {\bf Lemma 2} is false. A counterexample: $f(x) = f(x, 0)$. But trailing 0's are the only case---we can 
    see this via the proof of {\bf Lemma 2}. We can define $f^{-1}$ more carefully: 

    $$f^{-1}(a_1, \ldots, a_n) = f^{-1}(a_1, \ldots, a_i)$$

    where $a_i \neq 0$ for the largest $i \leq n$. $h$ can be saved with a modified projection function that returns 0 for
    indices greater than what exist. But $g$ is a goner.
\end{ans}


















\end{document}

